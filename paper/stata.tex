
% \section{User's guide to stata.sty}

% \texttt{stata.sty} is a {\LaTeX} package containing macros and environments to
% help authors produce documents containing {\stata} output and syntax
% diagrams.

% \subsection{Citing the Stata manuals}

% The macros for generating references to the {\stata} manuals are
% given in table~\ref{table:manref}.

% \clearpage
% \begin{table}[h!]
% \caption{Stata manual references}
% \label{table:manref}
% \begin{center}
% \begin{tabular}{ll}
% \hline
% \noalign{\smallskip}
% Example & Result\\ 
% \noalign{\smallskip}
% \hline
% \noalign{\smallskip}
% \verb+\bayesref{bayes}+ & \bayesref{bayes}\\
% \verb+\cmref{cmchoiceset}+ & \cmref{cmchoiceset} \\
% \verb+\dref{Data types}+ & \dref{Data types}\\
% \verb+\dsgeref{dsge}+ & \dsgeref{dsge} \\
% \verb+\ermref{eregress}+ & \ermref{eregress} \\
% \verb+\fnref{Statistical functions}+ & \fnref{Statistical functions}\\
% \verb+\fmmref{fmm:~betareg}+ & \fmmref{fmm:~betareg}\\
% \verb+\grefa{Graph Editor}+ & \grefa{Graph Editor}\\
% \verb+\grefb{graph}+ & \grefb{graph}\\
% \verb+\grefci{line\_options}+ & \grefci{line\_options}\\
% \verb+\grefdi{connectstyle}+ & \grefdi{connectstyle}\\
% \verb+\gsref{6~Using the Data Editor}+ & \gsref{6~Using the Data Editor} \\
% \verb+\irtref{irt}+ & \irtref{irt} \\
% \verb+\lassoref{Lasso intro}+ & \lassoref{Lasso intro}\\
% \verb+\metaref{meta}+ & \metaref{meta} \\
% \verb+\meref{me}+ & \meref{me}\\
% \verb+\mreff{Intro}+ & \mreff{Intro}\\
% \verb+\mrefa{Ado}+ & \mrefa{Ado}\\
% \verb+\mrefb{Declarations}+ & \mrefb{Declarations}\\
% \verb+\mrefc{mata clear}+ & \mrefc{mata clear}\\
% \verb+\mrefd{Matrix}+ & \mrefd{Matrix}\\
% \verb+\mrefe{st\_view($\,$)}+ & \mrefe{st\_view($\,$)}\\
% \verb+\mrefg{Glossary}+ & \mrefg{Glossary}\\
% \verb+\miref{mi impute}+ & \miref{mi impute}\\
% \verb+\mvref{cluster}+ & \mvref{cluster}\\
% \verb+\pref{syntax}+ & \pref{syntax}\\
% \verb+\pssrefa{Intro}+ & \pssrefa{Intro}\\
% \verb+\pssrefb{power}+ & \pssrefb{power}\\
% \verb+\pssrefc{ciwidth}+ & \pssrefc{ciwidth} \\
% \verb+\pssrefd{Unbalanced designs}+ & \pssrefd{Unbalanced designs}\\
% \verb+\pssrefe{Glossary}+ & \pssrefe{Glossary} \\
% \verb+\pssref{Subject and author index}+ & \pssref{Subject and author index} \\
% \verb+\rptref{Dynamic documents intro}+ & \rptref{Dynamic documents intro}\\
% \verb+\rref{regress}+ & \rref{regress}\\
% \verb+\spref{Intro}+ & \spref{Intro}\\
% \verb+\stref{streg}+ & \stref{streg}\\
% \verb+\svyref{svy:~tabulate oneway}+ & \svyref{svy:~tabulate oneway}\\
% \verb+\tsref{arima}+ & \tsref{arima}\\
% \verb+\uref{1~Read this---it will help}+ & \uref{1~Read this---it will help}\\
% \verb+\xtref{xtreg}+ & \xtref{xtreg}\\
% \noalign{\smallskip}
% \hline
% \end{tabular}
% \end{center}
% \end{table}

% \clearpage
% \subsection{Stata syntax}

% Here is an example syntax display:

% \begin{stsyntax}
% \dunderbar{reg}ress
%     \depvar\
%     \optindepvars\
%     \optif\
%     \optin\
%     \optweight\
%     \optional{,
%     \underbar{nocons}tant
%     \underbar{h}ascons
%     tsscons
%     vce({\it vcetype\/})
%     \underbar{l}evel(\num)
%     \underbar{b}eta
%     \underbar{ef}orm(\ststring)
%     \dunderbar{dep}name(\varname)
%     {\it display\_options}
%     \underbar{nohe}ader
%     \underbar{notab}le
%     plus
%     \underbar{ms}e1
%     \underbar{coefl}egend}
% \end{stsyntax}

% \noindent
% This syntax is generated by

% \begin{stverbatim}
% \begin{verbatim}
% \begin{stsyntax}
% \dunderbar{reg}ress
%     \depvar\
%     \optindepvars\
%     \optif\
%     \optin\
%     \optweight\
%     \optional{,
%     \underbar{nocons}tant
%     \underbar{h}ascons
%     tsscons
%     vce({\it vcetype\/})
%     \underbar{l}evel(\num)
%     \underbar{b}eta
%     \underbar{ef}orm(\ststring)
%     \dunderbar{dep}name(\varname)
%     {\it display\_options}
%     \underbar{nohe}ader
%     \underbar{notab}le
%     plus
%     \underbar{ms}e1
%     \underbar{coefl}egend}
% \end{stsyntax}
% \end{verbatim}
% \end{stverbatim}

% \noindent
% Each command should be formatted using a separate \texttt{stsyntax}
% environment.  Table~\ref{table:syntax} contains an example of each syntax
% macro provided in \texttt{stata.sty}. 

% \clearpage
% \begin{table}[h!]
% \caption{\stata{} syntax elements}
% \label{table:syntax}
% \fontsize{10}{14}\selectfont
% \begin{center}
% \begin{tabular}{ll@{\hspace{.5in}}ll}
% \noalign{\smallskip}
% \hline
% \noalign{\smallskip}
% Macro & Result
% &
% Macro & Result
% \\
% \noalign{\smallskip}
% \hline
% \noalign{\smallskip}
% \verb+\LB+ & \LB
% &
% \verb+\ifexp+ & \ifexp
% \\
% \noalign{\smallskip}
% \verb+\RB+ & \RB
% &
% \verb+\optif+ & \optif
% \\
% \noalign{\smallskip}
% \verb+\varname+ & \varname
% &
% \verb+\inrange+ & \inrange
% \\
% \noalign{\smallskip}
% \verb+\optvarname+ & \optvarname
% &
% \verb+\optin+ & \optin
% \\
% \noalign{\smallskip}
% \verb+\varlist+ & \varlist
% &
% \verb+\eqexp+ & \eqexp
% \\
% \noalign{\smallskip}
% \verb+\optvarlist+ & \optvarlist
% &
% \verb+\opteqexp+ & \opteqexp
% \\
% \noalign{\smallskip}
% \verb+\newvarname+ & \newvarname
% &
% \verb+\byvarlist+ & \byvarlist
% \\
% \noalign{\smallskip}
% \verb+\optnewvarname+ & \optnewvarname
% &
% \verb+\optby+ & \optby
% \\
% \noalign{\smallskip}
% \verb+\newvarlist+ & \newvarlist
% &
% \verb+\optional{text}+ & \optional{text}
% \\
% \noalign{\smallskip}
% \verb+\optnewvarlist+ & \optnewvarlist
% &
% \verb+\optweight+ & \optweight
% \\
% \noalign{\smallskip}
% \verb+\depvar+ & \depvar
% &
% \verb+\num+ & \num
% \\
% \noalign{\smallskip}
% \verb+\optindepvars+ & \optindepvars
% &
% \verb+\ststring+ & \ststring
% \\
% \noalign{\smallskip}
% \verb+\opttype+ & \opttype
% \\
% \noalign{\smallskip}
% \hline
% \end{tabular}
% \end{center}
% \end{table}

% \verb+\underbar+ is a standard macro that generates underlines.  The
% \verb+\dunderbar+ macro from \texttt{stata.sty} generates the underlines for
% words with descenders. For example,

% \begin{itemize}
% \item
% \verb+{\tt \underbar{reg}ress}+ generates {\tt \underbar{reg}ress}

% \item
% \verb+{\tt \dunderbar{reg}ress}+ generates {\tt \dunderbar{reg}ress}

% \end{itemize}

% The plain \TeX{} macros \verb+\it+, \verb+\sl+, and \verb+\tt+ are
% also available. \verb+\it+ should be used to denote ``replaceable''
% words, such as {\it varname}. \verb+\sl+ can be used for emphasis but
% should not be overused. \verb+\tt+ should be used to denote words that
% are to be typed, such as command names.

% When describing the options of a new command, the \verb+\hangpara+ and
% \verb+\morehang+ commands provide a means to reproduce a paragraph style
% similar to that of the Stata reference manuals.  For example,

% \hangpara
% {\tt level(\num)} specifies the confidence level, as a percentage,
% for confidence intervals.  The default is {\tt level(95)} or as set by {\tt
% set level}; see \uref{20.8~Specifying the width of confidence intervals}.

% \clearpage
% \noindent
% was generated by

% \begin{stverbatim}
% \begin{verbatim}
% \hangpara
% {\tt level(\num)} specifies the confidence level, as a percentage,
% for confidence intervals.  The default is {\tt level(95)} or as set by {\tt
% set level}; see \uref{20.8~Specifying the width of confidence intervals}.
% \end{verbatim}
% \end{stverbatim}


% \subsection{Stata output}
% \label{sec:output}

% When submitting {\sl Stata Journal\/} articles that contain {\stata} output,
% also submit a do-file and all relevant datasets that reproduce the output
% (do not forget to set the random-number seed when doing simulations).  Results
% should be reproducible. Begin examples by loading the data. Code should be
% written to respect a linesize of 80 characters. The
% following is an example of the \texttt{stlog} environment containing output
% from simple linear regression analysis on two variables in {\tt auto.dta}:

% \begin{stlog}
% \input{output1.log.tex}\nullskip
% \end{stlog}

% \noindent
% The above listing was included using

% \begin{stverbatim}
% \begin{verbatim}
% \begin{stlog}
% \input{output1.log.tex}\nullskip
% \end{stlog}
% \end{verbatim}
% \end{stverbatim}

% \noindent
% where \texttt{output1.log.tex} is a Stata log file converted to include \TeX{}
% macros by using the \stcmd{sjlog} command (more on \stcmd{sjlog} shortly).
% \verb+\nullskip+ adjusts the spacing around the log file.

% \clearpage
% On occasion, it is convenient (maybe even necessary) to be able to omit some of
% the output or let it spill onto the next page.  Here is a listing containing
% the details of the following discussion:

% \begin{stverbatim}
% \begin{verbatim}
% \begin{stlog}
% . sysuse auto
% (1978 Automobile Data)
% {\smallskip}
% . regress mpg weight
% {\smallskip}
% \oom
% {\smallskip}
% \clearpage
% \end{stlog}
% \end{verbatim}
% \end{stverbatim}

% The \verb+\oom+ macro creates
% a short message indicating omitted output in the following example, and the
% \verb+\clearpage+ macro inserts a page break.

% \begin{stlog}
% . sysuse auto
% (1978 Automobile Data)
% {\smallskip}
% . regress mpg weight
% {\smallskip}
% \oom
% {\smallskip}
% \clearpage
% \end{stlog}

% The output in \texttt{output1.log.tex} was generated from the following
% \texttt{output.do}:

% \begin{stlog}
% \input{output.do.log.tex}
% \end{stlog}

% \noindent
% \texttt{output.do} generates a \stcmd{.smcl} file, \stcmd{.log} file,
% and \stcmd{.log.tex} file using \stcmd{sjlog}.  The actual file used in the
% above listing was generated by

% \begin{stlog}
% . sjlog type output.do
% \end{stlog}

% \texttt{sjlog.ado} is provided in the Stata package for \stcmd{sjlatex}.
% \stcmd{sjlog} is a Stata command that helps generate log output to be included
% in {\LaTeX} documents using the \texttt{stlog} environment.  If you have
% installed the \stcmd{sjlatex} package, see the help file for \stcmd{sjlog} for
% more details.  The lines that make up the table output from \stcmd{regress}
% are generated from line-drawing macros defined in \texttt{stata.sty}; these
% were macros written using some font metrics defined in \citet{texbook}.

% By default, \texttt{stlog} sets an 8-point font for the log.  Use the
% \texttt{auto} option to turn this behavior off, allowing you to use the
% current font size, or change it by using\\ \verb+\fontsize{#}{#}\selectfont+.
% The call to \texttt{stlog} with the \texttt{auto} option looks like
% \verb+\begin[auto]{stlog}+.

% Here is an example where we are using a 12-point font.

% \vspace{-.2in}
% {\fontsize{12}{13}\selectfont
% \begin{stlog}[auto]
% . sjlog type output.do
% \end{stlog}
% }

% \subsection{About tables}

% Tables should be created using the standard \LaTeX{} methods.  See
% \citet{latexbook} for a discussion and examples. Tables should be included in
% the main text rather than at the end of the document. Tables should be called
% out in the text prior to appearance.

% \clearpage
% There are many user-written commands that produce \LaTeX{} output, including
% tables.  Christopher F. Baum has written \stcmd{outtable}, a Stata command for
% creating \LaTeX{} tables from Stata matrices.  Ben Jann's well-known
% \stcmd{estout} command can also produce \LaTeX{} output.  To find other
% user-written commands that produce \LaTeX{} output, try

% \begin{stlog}
% . net search latex
% \end{stlog}

% \subsubsection{Tables with notes}

% Table~\ref{Table4} shows the order and format to use for notes to tables.

% \begin{table}[h!]
% \begin{threeparttable}
% \centering
% \caption{Industrial clusters}
% \label{Table4}
% \begin{tabularx}{\textwidth}{X  p{1.5cm}  p{1pt}  X  p{1.5cm}}
% \hline
% \multicolumn{2}{c}{China} & & \multicolumn{2}{c}{United States} \\
% \cline{1-2} \cline{4-5}
% Core of cluster & Size (in $\num$ of units) & & Core of cluster & Size (in $\num$ of units) \\
% \cline{1-2} \cline{4-5}
% Construction & 28$^{\rm a}$ & & Public administration and defense; compulsory social
% security & 30$^{\rm b}$ \\
% Food, beverages, and tobacco & 3 & & Food, beverages, and tobacco & 2 \\
% Textiles and textile products & 2 & & Chemicals and chemical products & 1 \\
% Chemicals and chemical products & 1 & & Basic metals and fabricated metal & 1 \\
% Transport equipment & 1 & & Transport equipment & 1 \\
% \hline
% \multicolumn{2}{l}{$L_a=0.602$***} & & \multicolumn{2}{l}{$L_a=0.567$} \\
% \multicolumn{2}{l}{$L_w=0.828$**} & & \multicolumn{2}{l}{$L_w=0.837$} \\
% \multicolumn{2}{l}{$L_m=0.335$*} & & \multicolumn{2}{l}{$L_m=0.287$} \\
% \multicolumn{2}{l}{$K^*=5$} & & \multicolumn{2}{l}{$K^*=5$} \\
% \multicolumn{2}{l}{$K=35$} & & \multicolumn{2}{l}{$K=35$} \\
% \hline
% \end{tabularx}
% \begin{tablenotes}
% \footnotesize
% \item \textsc{source:} Pew Research Center.
% \item \textsc{note:} U.S.~industrial clusters based on
% U.S.~input--output flows of goods expressed in millions of dollars between 35
% {\smrm ISIC} industries from the {\smrm WIOD} data. The minimum number of
% clusters \texttt{k()} was set equal to five. The algorithm returns $L_a$,
% $L_w$, and $L_m$, which refer to the average of the internal relative
% flows, the population-weighted average of the internal relative
% flows, and the minimum of the internal relative flows,
% respectively. $K^*$ and $K$ refer to the number of defined regional clusters
% and the number of distinct starting units, respectively.
% \item $^{\rm a}$ This note pertains only to row 1 column 2.
% \item $^{\rm b}$ This note pertains only to row 1 column 4.
% \item *** denotes $p<0.01$; ** denotes $p<0.05$; * denotes $p<0.1$.
% \end{tablenotes}
% \end{threeparttable}
% \end{table}

% Order of notes should be
% \vspace{-.08in}
% \begin{enumerate}
% \item source notes
% \item notes applying to the whole table
% \item notes applying to specific parts of the table
% \item notes on significance levels
% \end{enumerate}

% Special notes:

% \vspace{-.08in}
% \begin{itemize}
% \item Use \verb+\centering+ because the {\tt center} environment adds
% unnecessary vertical spacing.

% \item Place the \verb+\begin{threeparttable}+ line above the caption.

% \end{itemize}

% Tables should be included in the main text rather than at the end of the
% document. Tables should be called out in the text prior to appearance.

% \subsection{Encapsulated PostScript (EPS)}

% You can include figures by using either \verb+\includegraphics+ or
% \verb+\epsfig+.

% \begin{stverbatim}
% \begin{verbatim}
% \begin{figure}[h!]
% \begin{center}
% \includegraphics{eps/output1.eps}
% \end{center}
% \caption{Scatterplot with simple linear regression line}
% \label{fig}
% \end{figure}
% \end{verbatim}
% \end{stverbatim}

% \begin{stverbatim}
% \begin{verbatim}
% \begin{figure}[h!]
% \begin{center}
% \epsfig{file=output1}
% \end{center}
% \caption{Scatterplot with simple linear regression line}
% \label{fig}
% \end{figure}
% \end{verbatim}
% \end{stverbatim}

% Figure~\ref{fig} is included using \verb+\epsfig+ from the \texttt{epsfig}
% package.  

% \noindent
% The graph was generated by running \texttt{output.do}, the
% do-file given in section~\ref{sec:output}.  The \texttt{epsfig} package is
% described in \citet*{latexcompanion}.

% \clearpage

% \begin{figure}[h!]
% \begin{center}
% \epsfig{file=output1}
% \end{center}
% \caption{Scatterplot with simple linear regression line}
% \label{fig}
% \end{figure}

% {\smrm EPS} is the preferred format for graphs and line art. Figures should be
% included in the main text rather than at the end of the document and should be
% called out in the text prior to appearance. If your article is written in
% Word, you should submit your figures as separate {\smrm EPS} files. Rasterized-based
% files of at least 300 dpi (dots per inch) are acceptable. Avoid using bitmaps
% for figures and graphs, because even if images are outputted at 300 dpi,
% bitmaps can increase the size of the resulting file for printing. (However,
% bitmaps will be allowed for photographs, which are used in, for example, the
% {\sl Stata Journal} Editors' prize announcement.) Images should be submitted in
% black and white (grayscale). We recommend that graphs created in Stata use the
% {\tt sj} scheme.

% \subsection{Stored results}

% The \texttt{stresults} environment provides a table to describe the stored
% results of a Stata command.  It consists of four columns: the first and third
% column are for Stata result identifiers (for example, \stcmd{r(N)},
% \stcmd{e(cmd)}), and the second and fourth columns are for a brief description
% of the respective identifier.
% %
% Each group of results is generated using the \verb+\stresultsgroup+ macro.
% %
% The following is an example containing a brief description of the results that
% \stcmd{regress} stored to \stcmd{e()}:

% \clearpage

% \begin{stresults}
% \stresultsgroup{Scalars} \\
% \stcmd{e(N)} & number of observations
% &
% \stcmd{e(F)} & $\scriptstyle F$ statistic
% \\
% \stcmd{e(mss)} & model sum of squares
% &
% \stcmd{e(rmse)} & root mean squared error
% \\
% \stcmd{e(df\_m)} & model degrees of freedom
% &
% \stcmd{e(ll\_r)} & log likelihood
% \\
% \stcmd{e(rss)} & residual sum of squares
% &
% \stcmd{e(ll\_r0)} & log likelihood, constant-only\\
% \stcmd{e(df\_r)} & residual degrees of freedom
% &
% & \quad model \\
% \stcmd{e(r2)} & $\scriptstyle R$-squared
% &
% \stcmd{e(N\_clust)} & number of clusters
% \\
% \stresultsgroup{Macros} \\
% \stcmd{e(cmd)} & \stcmd{regress}
% &
% \stcmd{e(wexp)} & weight expression
% \\
% \stcmd{e(depvar)} & name of dependent variable
% &
% \stcmd{e(clustvar)} & name of cluster variable
% \\
% \stcmd{e(model)} & \stcmd{ols} or \stcmd{iv}
% &
% \stcmd{e(vcetype)} & title used to label Std.~Err.
% \\
% \stcmd{e(wtype)} & weight type
% &
% \stcmd{e(predict)} & program used to implement
% \\
% &&&\quad \stcmd{predict}
% \\
% \stresultsgroup{Matrices} \\
% \stcmd{e(b)} & coefficient vector
% &
% \stcmd{e(V)} & variance--covariance matrix of
% \\
% &&&\quad the estimators\\
% \stresultsgroup{Functions} \\
% \stcmd{e(sample)} & marks estimation sample
% \\
% \end{stresults}

% Alternatively, you can use the \texttt{stresults2} environment to create
% a two column table.   This format works better if your descriptions are long.

% \subsection{Examples and notes}

% The following are environments for examples and notes similar to those
% given in the Stata reference manuals.  They are generated using the
% \texttt{stexample} and \texttt{sttech} environments, respectively.


% \begin{stexample}
% This is the default alignment for a \stata{} example.
% \end{stexample}

% \setlength{\stexamplehskip}{0pt}
% \begin{stexample}
% For this example, \verb+\stexamplehskip+ was set to
% \texttt{\the\stexamplehskip}
% before beginning.  This sentence is supposed to spill
% over to the next line, thus revealing that the first sentence was indented.

% This sentence is supposed to show that new paragraphs are automatically
% indented (provided that \verb+\parindent+ is nonzero).
% \end{stexample}

% \clearpage
% \begin{sttech}
% For this note, \verb+\sttechhskip+ was set to \texttt{\the\sttechhskip}
% (the default) before beginning.  This sentence is supposed to spill over to
% the next line, thus revealing that the first sentence was indented.

% This sentence is supposed to show that new paragraphs are automatically
% indented (provided that \verb+\parindent+ is nonzero).
% \end{sttech}

% \subsection{Special characters}

% Table \ref{table:specialch} contains macros that generate some useful
% characters in the typewriter (fixed width) font.  The exceptions are
% \verb+\stcaret+ and \verb+\sttilde+, which use the currently specified font;
% the strictly fixed-width versions are \verb+\caret+ and \verb+\tytilde+,
% respectively.

% \begin{table}[h!]
% \caption{Special characters}
% \label{table:specialch}
% \begin{center}
% \begin{tabular}{ll@{\hspace{.5in}}ll}
% \hline
% \noalign{\smallskip}
% Macro & Result &
% Macro & Result \\ 
% \noalign{\smallskip}
% \hline
% \noalign{\smallskip}
% \verb+\stbackslash+ & \stbackslash
%  &
% \verb+\sttilde+ & \sttilde
% \\
% \verb+\stforslash+ & \stforslash 
% &
% \verb+\tytilde+ & \tytilde
% \\
% \verb+\stcaret+ & \stcaret
% &
% \verb+\lbr+ & \lbr
% \\
% \verb+\caret+ & \caret
% &
% \verb+\rbr+ & \rbr
% \\
% \noalign{\smallskip}
% \hline
% \end{tabular}
% \end{center}
% \end{table}


% \subsection{Equations and formulas}

% In (\ref{eq:Exbar}), $\stbar{x}$ was generated using
% \verb+\stbar{x}+.  Here \verb+\stbar+ is equivalent to the \TeX{} macro
% \verb+\overline+.

% \begin{equation}
% E(\stbar{x}) = \mu
% \label{eq:Exbar}
% \end{equation}

% In (\ref{eq:varbetahat}), $\sthat{\beta}$ was generated using
% \verb+\sthat{\beta}+.  Here \verb+\sthat+ is equivalent to the \TeX{} macro
% \verb+\widehat+.

% \begin{equation}
% V(\sthat{\beta}) = V\{(X'X)^{-1}X'y\} = (X'X)^{-1}X'V(y)X(X'X)^{-1}
% \label{eq:varbetahat}
% \end{equation}

% \clearpage

% Formulas should be defined and follow a concise style. Different
% disciplines adhere to different notation styles; however, if the
% notation cannot be clearly interpreted, you may be asked to make
% changes. The bolding and font selection guidelines are the following:

% \begin{itemize}
% \item
% 	Matrices are capitalized and bolded; for instance, $\boldsymbol\Pi +
% 	\boldsymbol\Theta + \boldsymbol\Phi - \mathbf{B}$.

% \item
% 	Vectors are lowercased and bolded; for instance, $\boldsymbol\pi +
% 	\boldsymbol\theta + \boldsymbol\phi - \mathbf{b}$.

% \item
%         Scalars are lowercased and nonbolded; for instance, $r_2 + c_1 - c_2$.
% \end{itemize}


% Sentence punctuation should not be used in formulas set off from the text.

% Formulas in line with the text should use the solidus (/) instead of a
% horizontal line for fractional terms.

% Nesting of grouping is square brackets, curly braces, and then parentheses, or
% [\{()\}].

% Only those equations explicitly referred to in the text should be assigned an
% equation number.

% %\subsection{Other miscellaneous macros and environments}
% %
% %The following box was created by
% %
% %\begin{stverbatim}
% %\begin{verbatim}
% %\fbox{
% %A special framed
% %box that obeys lines and spaces.
% %}
% %\end{verbatim}
% %\end{stverbatim}
% %
% %\fbox{
% %A special framed
% %box that obeys lines and spaces.
% %}
% %
% %The following box was created by
% %
% %\begin{stverbatim}
% %\begin{verbatim}
% %\fbox{
% %Test that the width of the
% %box is \the\ttboxWd
% %and is indented \the\ttboxIndent
% %}
% %\end{verbatim}
% %\end{stverbatim}
% %
% %\fbox{
% %Test that the width of the
% %box is \the\ttboxWd
% %and is indented \the\ttboxIndent
% %}
% %
% \endinput
